% Options for packages loaded elsewhere
\PassOptionsToPackage{unicode}{hyperref}
\PassOptionsToPackage{hyphens}{url}
%
\documentclass[
]{article}
\usepackage{amsmath,amssymb}
\usepackage{lmodern}
\usepackage{iftex}
\ifPDFTeX
  \usepackage[T1]{fontenc}
  \usepackage[utf8]{inputenc}
  \usepackage{textcomp} % provide euro and other symbols
\else % if luatex or xetex
  \usepackage{unicode-math}
  \defaultfontfeatures{Scale=MatchLowercase}
  \defaultfontfeatures[\rmfamily]{Ligatures=TeX,Scale=1}
\fi
% Use upquote if available, for straight quotes in verbatim environments
\IfFileExists{upquote.sty}{\usepackage{upquote}}{}
\IfFileExists{microtype.sty}{% use microtype if available
  \usepackage[]{microtype}
  \UseMicrotypeSet[protrusion]{basicmath} % disable protrusion for tt fonts
}{}
\makeatletter
\@ifundefined{KOMAClassName}{% if non-KOMA class
  \IfFileExists{parskip.sty}{%
    \usepackage{parskip}
  }{% else
    \setlength{\parindent}{0pt}
    \setlength{\parskip}{6pt plus 2pt minus 1pt}}
}{% if KOMA class
  \KOMAoptions{parskip=half}}
\makeatother
\usepackage{xcolor}
\usepackage[margin=1in]{geometry}
\usepackage{color}
\usepackage{fancyvrb}
\newcommand{\VerbBar}{|}
\newcommand{\VERB}{\Verb[commandchars=\\\{\}]}
\DefineVerbatimEnvironment{Highlighting}{Verbatim}{commandchars=\\\{\}}
% Add ',fontsize=\small' for more characters per line
\usepackage{framed}
\definecolor{shadecolor}{RGB}{248,248,248}
\newenvironment{Shaded}{\begin{snugshade}}{\end{snugshade}}
\newcommand{\AlertTok}[1]{\textcolor[rgb]{0.94,0.16,0.16}{#1}}
\newcommand{\AnnotationTok}[1]{\textcolor[rgb]{0.56,0.35,0.01}{\textbf{\textit{#1}}}}
\newcommand{\AttributeTok}[1]{\textcolor[rgb]{0.77,0.63,0.00}{#1}}
\newcommand{\BaseNTok}[1]{\textcolor[rgb]{0.00,0.00,0.81}{#1}}
\newcommand{\BuiltInTok}[1]{#1}
\newcommand{\CharTok}[1]{\textcolor[rgb]{0.31,0.60,0.02}{#1}}
\newcommand{\CommentTok}[1]{\textcolor[rgb]{0.56,0.35,0.01}{\textit{#1}}}
\newcommand{\CommentVarTok}[1]{\textcolor[rgb]{0.56,0.35,0.01}{\textbf{\textit{#1}}}}
\newcommand{\ConstantTok}[1]{\textcolor[rgb]{0.00,0.00,0.00}{#1}}
\newcommand{\ControlFlowTok}[1]{\textcolor[rgb]{0.13,0.29,0.53}{\textbf{#1}}}
\newcommand{\DataTypeTok}[1]{\textcolor[rgb]{0.13,0.29,0.53}{#1}}
\newcommand{\DecValTok}[1]{\textcolor[rgb]{0.00,0.00,0.81}{#1}}
\newcommand{\DocumentationTok}[1]{\textcolor[rgb]{0.56,0.35,0.01}{\textbf{\textit{#1}}}}
\newcommand{\ErrorTok}[1]{\textcolor[rgb]{0.64,0.00,0.00}{\textbf{#1}}}
\newcommand{\ExtensionTok}[1]{#1}
\newcommand{\FloatTok}[1]{\textcolor[rgb]{0.00,0.00,0.81}{#1}}
\newcommand{\FunctionTok}[1]{\textcolor[rgb]{0.00,0.00,0.00}{#1}}
\newcommand{\ImportTok}[1]{#1}
\newcommand{\InformationTok}[1]{\textcolor[rgb]{0.56,0.35,0.01}{\textbf{\textit{#1}}}}
\newcommand{\KeywordTok}[1]{\textcolor[rgb]{0.13,0.29,0.53}{\textbf{#1}}}
\newcommand{\NormalTok}[1]{#1}
\newcommand{\OperatorTok}[1]{\textcolor[rgb]{0.81,0.36,0.00}{\textbf{#1}}}
\newcommand{\OtherTok}[1]{\textcolor[rgb]{0.56,0.35,0.01}{#1}}
\newcommand{\PreprocessorTok}[1]{\textcolor[rgb]{0.56,0.35,0.01}{\textit{#1}}}
\newcommand{\RegionMarkerTok}[1]{#1}
\newcommand{\SpecialCharTok}[1]{\textcolor[rgb]{0.00,0.00,0.00}{#1}}
\newcommand{\SpecialStringTok}[1]{\textcolor[rgb]{0.31,0.60,0.02}{#1}}
\newcommand{\StringTok}[1]{\textcolor[rgb]{0.31,0.60,0.02}{#1}}
\newcommand{\VariableTok}[1]{\textcolor[rgb]{0.00,0.00,0.00}{#1}}
\newcommand{\VerbatimStringTok}[1]{\textcolor[rgb]{0.31,0.60,0.02}{#1}}
\newcommand{\WarningTok}[1]{\textcolor[rgb]{0.56,0.35,0.01}{\textbf{\textit{#1}}}}
\usepackage{graphicx}
\makeatletter
\def\maxwidth{\ifdim\Gin@nat@width>\linewidth\linewidth\else\Gin@nat@width\fi}
\def\maxheight{\ifdim\Gin@nat@height>\textheight\textheight\else\Gin@nat@height\fi}
\makeatother
% Scale images if necessary, so that they will not overflow the page
% margins by default, and it is still possible to overwrite the defaults
% using explicit options in \includegraphics[width, height, ...]{}
\setkeys{Gin}{width=\maxwidth,height=\maxheight,keepaspectratio}
% Set default figure placement to htbp
\makeatletter
\def\fps@figure{htbp}
\makeatother
\setlength{\emergencystretch}{3em} % prevent overfull lines
\providecommand{\tightlist}{%
  \setlength{\itemsep}{0pt}\setlength{\parskip}{0pt}}
\setcounter{secnumdepth}{-\maxdimen} % remove section numbering
\ifLuaTeX
  \usepackage{selnolig}  % disable illegal ligatures
\fi
\IfFileExists{bookmark.sty}{\usepackage{bookmark}}{\usepackage{hyperref}}
\IfFileExists{xurl.sty}{\usepackage{xurl}}{} % add URL line breaks if available
\urlstyle{same} % disable monospaced font for URLs
\hypersetup{
  pdftitle={PR2},
  pdfauthor={강현승},
  hidelinks,
  pdfcreator={LaTeX via pandoc}}

\title{PR2}
\author{강현승}
\date{2022-09-15}

\begin{document}
\maketitle

\hypertarget{ruxb85c-uxacc4uxc0b0uxd558uxae30}{%
\section{1. R로 계산하기}\label{ruxb85c-uxacc4uxc0b0uxd558uxae30}}

\hypertarget{uxae30uxbcf8uxc5f0uxc0b0}{%
\subsection{1.1 기본연산}\label{uxae30uxbcf8uxc5f0uxc0b0}}

\begin{Shaded}
\begin{Highlighting}[]
\DecValTok{31} \SpecialCharTok{+} \DecValTok{3}
\end{Highlighting}
\end{Shaded}

\begin{verbatim}
## [1] 34
\end{verbatim}

\begin{Shaded}
\begin{Highlighting}[]
\DecValTok{15} \SpecialCharTok{{-}} \DecValTok{3} \SpecialCharTok{+} \DecValTok{7}
\end{Highlighting}
\end{Shaded}

\begin{verbatim}
## [1] 19
\end{verbatim}

\begin{Shaded}
\begin{Highlighting}[]
\DecValTok{13} \SpecialCharTok{*} \DecValTok{2} \SpecialCharTok{{-}} \DecValTok{6} \SpecialCharTok{/} \DecValTok{2}
\end{Highlighting}
\end{Shaded}

\begin{verbatim}
## [1] 23
\end{verbatim}

\begin{Shaded}
\begin{Highlighting}[]
\DecValTok{13} \SpecialCharTok{*}\NormalTok{ (}\DecValTok{2} \SpecialCharTok{{-}} \DecValTok{6}\NormalTok{) }\SpecialCharTok{/} \DecValTok{2}
\end{Highlighting}
\end{Shaded}

\begin{verbatim}
## [1] -26
\end{verbatim}

\begin{Shaded}
\begin{Highlighting}[]
\DecValTok{8} \SpecialCharTok{\%/\%} \DecValTok{2}
\end{Highlighting}
\end{Shaded}

\begin{verbatim}
## [1] 4
\end{verbatim}

\begin{Shaded}
\begin{Highlighting}[]
\DecValTok{11} \SpecialCharTok{\%\%} \DecValTok{3}
\end{Highlighting}
\end{Shaded}

\begin{verbatim}
## [1] 2
\end{verbatim}

\begin{Shaded}
\begin{Highlighting}[]
\NormalTok{n }\OtherTok{=} \DecValTok{21} \SpecialCharTok{\%\%} \DecValTok{4}
\FunctionTok{print}\NormalTok{(n)}
\end{Highlighting}
\end{Shaded}

\begin{verbatim}
## [1] 1
\end{verbatim}

\hypertarget{uxc218uxd559uxd568uxc218-uxc0acuxc6a9}{%
\subsection{1.2 수학함수
사용}\label{uxc218uxd559uxd568uxc218-uxc0acuxc6a9}}

\begin{Shaded}
\begin{Highlighting}[]
\FunctionTok{log}\NormalTok{(}\DecValTok{2}\NormalTok{)}
\end{Highlighting}
\end{Shaded}

\begin{verbatim}
## [1] 0.6931472
\end{verbatim}

\begin{Shaded}
\begin{Highlighting}[]
\FunctionTok{log}\NormalTok{(}\FunctionTok{exp}\NormalTok{(}\DecValTok{2}\NormalTok{))}
\end{Highlighting}
\end{Shaded}

\begin{verbatim}
## [1] 2
\end{verbatim}

\begin{Shaded}
\begin{Highlighting}[]
\FunctionTok{sqrt}\NormalTok{(}\DecValTok{4}\NormalTok{)}
\end{Highlighting}
\end{Shaded}

\begin{verbatim}
## [1] 2
\end{verbatim}

\begin{Shaded}
\begin{Highlighting}[]
\DecValTok{4} \SpecialCharTok{\^{}} \DecValTok{5}
\end{Highlighting}
\end{Shaded}

\begin{verbatim}
## [1] 1024
\end{verbatim}

\begin{Shaded}
\begin{Highlighting}[]
\DecValTok{4} \SpecialCharTok{**} \DecValTok{5}
\end{Highlighting}
\end{Shaded}

\begin{verbatim}
## [1] 1024
\end{verbatim}

\begin{Shaded}
\begin{Highlighting}[]
\FunctionTok{round}\NormalTok{(}\FloatTok{9.13}\NormalTok{)}
\end{Highlighting}
\end{Shaded}

\begin{verbatim}
## [1] 9
\end{verbatim}

\begin{Shaded}
\begin{Highlighting}[]
\FunctionTok{ceiling}\NormalTok{(}\FloatTok{1.41}\NormalTok{)}
\end{Highlighting}
\end{Shaded}

\begin{verbatim}
## [1] 2
\end{verbatim}

\begin{Shaded}
\begin{Highlighting}[]
\FunctionTok{floor}\NormalTok{(}\FloatTok{1.95}\NormalTok{)}
\end{Highlighting}
\end{Shaded}

\begin{verbatim}
## [1] 1
\end{verbatim}

\begin{Shaded}
\begin{Highlighting}[]
\NormalTok{pi}
\end{Highlighting}
\end{Shaded}

\begin{verbatim}
## [1] 3.141593
\end{verbatim}

\hypertarget{uxc218uxce58-uxc694uxc57duxd558uxae30}{%
\section{2. 수치 요약하기}\label{uxc218uxce58-uxc694uxc57duxd558uxae30}}

\hypertarget{uxbca1uxd130-uxc0dduxc131-uxbc0f-uxcd9cuxb825}{%
\subsection{2.1 벡터 생성 및
출력}\label{uxbca1uxd130-uxc0dduxc131-uxbc0f-uxcd9cuxb825}}

\begin{itemize}
\tightlist
\item
  정수형 값이 저장된 벡터를 생성하기
\item
  벡터 출력해보기
\end{itemize}

\begin{Shaded}
\begin{Highlighting}[]
\NormalTok{v1 }\OtherTok{=} \DecValTok{3}
\NormalTok{v2 }\OtherTok{=} \FunctionTok{c}\NormalTok{(}\DecValTok{4}\NormalTok{, }\DecValTok{5}\NormalTok{)}
\NormalTok{v3 }\OtherTok{=} \DecValTok{3}\SpecialCharTok{:}\DecValTok{11}
\NormalTok{v4 }\OtherTok{=} \FunctionTok{c}\NormalTok{(v1, v2, v3)}
\FunctionTok{print}\NormalTok{(v1)}
\end{Highlighting}
\end{Shaded}

\begin{verbatim}
## [1] 3
\end{verbatim}

\begin{Shaded}
\begin{Highlighting}[]
\FunctionTok{print}\NormalTok{(v2)}
\end{Highlighting}
\end{Shaded}

\begin{verbatim}
## [1] 4 5
\end{verbatim}

\begin{Shaded}
\begin{Highlighting}[]
\FunctionTok{print}\NormalTok{(v3)}
\end{Highlighting}
\end{Shaded}

\begin{verbatim}
## [1]  3  4  5  6  7  8  9 10 11
\end{verbatim}

\begin{Shaded}
\begin{Highlighting}[]
\FunctionTok{print}\NormalTok{(v4)}
\end{Highlighting}
\end{Shaded}

\begin{verbatim}
##  [1]  3  4  5  3  4  5  6  7  8  9 10 11
\end{verbatim}

\begin{Shaded}
\begin{Highlighting}[]
\NormalTok{v1 }\SpecialCharTok{*} \DecValTok{2}
\end{Highlighting}
\end{Shaded}

\begin{verbatim}
## [1] 6
\end{verbatim}

\begin{Shaded}
\begin{Highlighting}[]
\NormalTok{v1 }\SpecialCharTok{/}\NormalTok{ v3}
\end{Highlighting}
\end{Shaded}

\begin{verbatim}
## [1] 1.0000000 0.7500000 0.6000000 0.5000000 0.4285714 0.3750000 0.3333333
## [8] 0.3000000 0.2727273
\end{verbatim}

\hypertarget{uxd3c9uxade0uxad6cuxd558uxae30}{%
\subsection{2.2 평균구하기}\label{uxd3c9uxade0uxad6cuxd558uxae30}}

\begin{itemize}
\tightlist
\item
  평균을 구하는 여러가지 방법
\end{itemize}

\begin{Shaded}
\begin{Highlighting}[]
\NormalTok{(}\DecValTok{1} \SpecialCharTok{+} \DecValTok{2} \SpecialCharTok{+} \DecValTok{3} \SpecialCharTok{+} \DecValTok{4} \SpecialCharTok{+} \DecValTok{5} \SpecialCharTok{+} \DecValTok{6} \SpecialCharTok{+} \DecValTok{7} \SpecialCharTok{+} \DecValTok{8} \SpecialCharTok{+} \DecValTok{9}\NormalTok{)}\SpecialCharTok{/}\DecValTok{9}
\end{Highlighting}
\end{Shaded}

\begin{verbatim}
## [1] 5
\end{verbatim}

\begin{Shaded}
\begin{Highlighting}[]
\FunctionTok{sum}\NormalTok{(}\DecValTok{1}\NormalTok{,}\DecValTok{2}\NormalTok{,}\DecValTok{3}\NormalTok{,}\DecValTok{4}\NormalTok{,}\DecValTok{5}\NormalTok{,}\DecValTok{6}\NormalTok{,}\DecValTok{7}\NormalTok{,}\DecValTok{8}\NormalTok{,}\DecValTok{9}\NormalTok{)}\SpecialCharTok{/}\DecValTok{9}
\end{Highlighting}
\end{Shaded}

\begin{verbatim}
## [1] 5
\end{verbatim}

\begin{Shaded}
\begin{Highlighting}[]
\NormalTok{v5 }\OtherTok{=} \DecValTok{1}\SpecialCharTok{:}\DecValTok{9}
\FunctionTok{sum}\NormalTok{(v5) }\SpecialCharTok{/} \FunctionTok{length}\NormalTok{(v5)}
\end{Highlighting}
\end{Shaded}

\begin{verbatim}
## [1] 5
\end{verbatim}

\begin{Shaded}
\begin{Highlighting}[]
\FunctionTok{mean}\NormalTok{(v5)}
\end{Highlighting}
\end{Shaded}

\begin{verbatim}
## [1] 5
\end{verbatim}

\hypertarget{uxd568uxc218uxd65cuxc6a9}{%
\subsection{2.3 함수활용}\label{uxd568uxc218uxd65cuxc6a9}}

\begin{Shaded}
\begin{Highlighting}[]
\FunctionTok{mean}\NormalTok{(v5)}
\end{Highlighting}
\end{Shaded}

\begin{verbatim}
## [1] 5
\end{verbatim}

\begin{Shaded}
\begin{Highlighting}[]
\FunctionTok{var}\NormalTok{(v5)}
\end{Highlighting}
\end{Shaded}

\begin{verbatim}
## [1] 7.5
\end{verbatim}

\begin{Shaded}
\begin{Highlighting}[]
\FunctionTok{sd}\NormalTok{(v5)}
\end{Highlighting}
\end{Shaded}

\begin{verbatim}
## [1] 2.738613
\end{verbatim}

\begin{Shaded}
\begin{Highlighting}[]
\FunctionTok{median}\NormalTok{(v5)}
\end{Highlighting}
\end{Shaded}

\begin{verbatim}
## [1] 5
\end{verbatim}

\begin{Shaded}
\begin{Highlighting}[]
\FunctionTok{max}\NormalTok{(v5)}
\end{Highlighting}
\end{Shaded}

\begin{verbatim}
## [1] 9
\end{verbatim}

\begin{Shaded}
\begin{Highlighting}[]
\FunctionTok{min}\NormalTok{(v5)}
\end{Highlighting}
\end{Shaded}

\begin{verbatim}
## [1] 1
\end{verbatim}

\begin{Shaded}
\begin{Highlighting}[]
\NormalTok{v6 }\OtherTok{=} \DecValTok{1}\SpecialCharTok{:}\DecValTok{10}
\FunctionTok{median}\NormalTok{(v6)}
\end{Highlighting}
\end{Shaded}

\begin{verbatim}
## [1] 5.5
\end{verbatim}

\hypertarget{uxbb38uxc790uxac12uxc774-uxc800uxc7a5uxb41c-uxbca1uxd130-uxc0dduxc131}{%
\section{3. 문자값이 저장된 벡터
생성}\label{uxbb38uxc790uxac12uxc774-uxc800uxc7a5uxb41c-uxbca1uxd130-uxc0dduxc131}}

\begin{Shaded}
\begin{Highlighting}[]
\NormalTok{myEmail }\OtherTok{=} \StringTok{"h5k@ajou.ac.kr"}
\NormalTok{birthday }\OtherTok{=} \FunctionTok{c}\NormalTok{(}\StringTok{"2002년"}\NormalTok{, }\StringTok{"5월"}\NormalTok{, }\StringTok{"13일"}\NormalTok{)}
\NormalTok{birthday2 }\OtherTok{=} \FunctionTok{paste}\NormalTok{(}\StringTok{"2002년"}\NormalTok{, }\StringTok{"5월"}\NormalTok{, }\StringTok{"13일"}\NormalTok{)}
\NormalTok{birthday3 }\OtherTok{=} \FunctionTok{paste0}\NormalTok{(}\StringTok{"2002년"}\NormalTok{, }\StringTok{"5월"}\NormalTok{, }\StringTok{"13일"}\NormalTok{)}

\FunctionTok{print}\NormalTok{(myEmail)}
\end{Highlighting}
\end{Shaded}

\begin{verbatim}
## [1] "h5k@ajou.ac.kr"
\end{verbatim}

\begin{Shaded}
\begin{Highlighting}[]
\FunctionTok{print}\NormalTok{(birthday)}
\end{Highlighting}
\end{Shaded}

\begin{verbatim}
## [1] "2002년" "5월"    "13일"
\end{verbatim}

\begin{Shaded}
\begin{Highlighting}[]
\FunctionTok{print}\NormalTok{(birthday2)}
\end{Highlighting}
\end{Shaded}

\begin{verbatim}
## [1] "2002년 5월 13일"
\end{verbatim}

\begin{Shaded}
\begin{Highlighting}[]
\FunctionTok{print}\NormalTok{(birthday3)}
\end{Highlighting}
\end{Shaded}

\begin{verbatim}
## [1] "2002년5월13일"
\end{verbatim}

\hypertarget{uxae30uxd0c0-r-uxd65cuxc6a9uxbc95}{%
\section{4. 기타 R 활용법}\label{uxae30uxd0c0-r-uxd65cuxc6a9uxbc95}}

\hypertarget{uxc138uxbbf8uxcf5cuxb860-uxd65cuxc6a9-uxbc0f-uxbcc0uxc218uxba85uxb9ccuxc73cuxb85c-uxcd9cuxb825uxd558uxae30}{%
\subsection{4.1 세미콜론 활용 및 변수명만으로
출력하기}\label{uxc138uxbbf8uxcf5cuxb860-uxd65cuxc6a9-uxbc0f-uxbcc0uxc218uxba85uxb9ccuxc73cuxb85c-uxcd9cuxb825uxd558uxae30}}

\begin{Shaded}
\begin{Highlighting}[]
\FunctionTok{mean}\NormalTok{(v5)}
\end{Highlighting}
\end{Shaded}

\begin{verbatim}
## [1] 5
\end{verbatim}

\begin{Shaded}
\begin{Highlighting}[]
\FunctionTok{sd}\NormalTok{(v5);}\FunctionTok{plot}\NormalTok{(v5)}
\end{Highlighting}
\end{Shaded}

\begin{verbatim}
## [1] 2.738613
\end{verbatim}

\includegraphics{pr2_202221498_강현승_files/figure-latex/unnamed-chunk-7-1.pdf}

\begin{Shaded}
\begin{Highlighting}[]
\NormalTok{myEmail}
\end{Highlighting}
\end{Shaded}

\begin{verbatim}
## [1] "h5k@ajou.ac.kr"
\end{verbatim}

\begin{Shaded}
\begin{Highlighting}[]
\NormalTok{birthday}
\end{Highlighting}
\end{Shaded}

\begin{verbatim}
## [1] "2002년" "5월"    "13일"
\end{verbatim}

\hypertarget{uxc791uxc5c5uxd3f4uxb354-uxd655uxc778-uxbc0f-uxbcc0uxacbd}{%
\subsection{4.2 작업폴더 확인 및
변경}\label{uxc791uxc5c5uxd3f4uxb354-uxd655uxc778-uxbc0f-uxbcc0uxacbd}}

\begin{itemize}
\tightlist
\item
  변경할 폴더는 사전에 만들어져 있는 폴더여야함
\item
  본인이 작업할 폴더의 경로를 setwd(``\,``) 의 따옴표 사이에 입력
\item
  작업할 폴더는 본인이 원하는 경로로 지정해주세요.
\item
  예) setwd(``c:/data'')
\end{itemize}

\begin{Shaded}
\begin{Highlighting}[]
\FunctionTok{getwd}\NormalTok{()}
\end{Highlighting}
\end{Shaded}

\begin{verbatim}
## [1] "/Users/hyeonseungkang/RProjects/ajou-r-programming/pr2"
\end{verbatim}

\begin{Shaded}
\begin{Highlighting}[]
\FunctionTok{setwd}\NormalTok{(}\StringTok{"/Users/hyeonseungkang/RProjects"}\NormalTok{)}
\FunctionTok{getwd}\NormalTok{()}
\end{Highlighting}
\end{Shaded}

\begin{verbatim}
## [1] "/Users/hyeonseungkang/RProjects"
\end{verbatim}

\hypertarget{pr2-uxc5f0uxc2b5uxbb38uxc81c}{%
\section{PR2 연습문제}\label{pr2-uxc5f0uxc2b5uxbb38uxc81c}}

\hypertarget{uxbb38uxc81c-1}{%
\subsection{문제 1}\label{uxbb38uxc81c-1}}

\begin{Shaded}
\begin{Highlighting}[]
\CommentTok{\# q1}

\NormalTok{student\_num }\OtherTok{=} \FunctionTok{c}\NormalTok{(}\DecValTok{11}\NormalTok{, }\DecValTok{15}\NormalTok{, }\DecValTok{3}\NormalTok{, }\DecValTok{8}\NormalTok{, }\DecValTok{6}\NormalTok{, }\DecValTok{6}\NormalTok{, }\DecValTok{8}\NormalTok{, }\DecValTok{13}\NormalTok{)}
\NormalTok{x.scaled }\OtherTok{=}\NormalTok{ (student\_num }\SpecialCharTok{{-}} \FunctionTok{min}\NormalTok{(student\_num)) }\SpecialCharTok{/}\NormalTok{ (}\FunctionTok{max}\NormalTok{(student\_num) }\SpecialCharTok{{-}} \FunctionTok{min}\NormalTok{(student\_num))}
\NormalTok{x.scaled}
\end{Highlighting}
\end{Shaded}

\begin{verbatim}
## [1] 0.6666667 1.0000000 0.0000000 0.4166667 0.2500000 0.2500000 0.4166667
## [8] 0.8333333
\end{verbatim}

\hypertarget{uxbb38uxc81c-2}{%
\subsection{문제 2}\label{uxbb38uxc81c-2}}

\begin{Shaded}
\begin{Highlighting}[]
\CommentTok{\# q2}

\NormalTok{NIR }\OtherTok{=} \DecValTok{60}
\NormalTok{RED }\OtherTok{=} \DecValTok{26}
\NormalTok{NDVI }\OtherTok{=}\NormalTok{ (NIR }\SpecialCharTok{{-}}\NormalTok{ RED) }\SpecialCharTok{/}\NormalTok{ (NIR }\SpecialCharTok{+}\NormalTok{ RED)}
\NormalTok{NDVI}
\end{Highlighting}
\end{Shaded}

\begin{verbatim}
## [1] 0.3953488
\end{verbatim}

\hypertarget{uxbb38uxc81c-3}{%
\subsection{문제 3}\label{uxbb38uxc81c-3}}

\begin{Shaded}
\begin{Highlighting}[]
\CommentTok{\# q3}

\NormalTok{n }\OtherTok{=} \DecValTok{12}

\NormalTok{undirected }\OtherTok{=}\NormalTok{ n }\SpecialCharTok{*}\NormalTok{ (n }\SpecialCharTok{{-}} \DecValTok{1}\NormalTok{) }\SpecialCharTok{/} \DecValTok{2}
\NormalTok{directed }\OtherTok{=}\NormalTok{ n }\SpecialCharTok{*}\NormalTok{ (n }\SpecialCharTok{{-}} \DecValTok{1}\NormalTok{)}

\FunctionTok{print}\NormalTok{(undirected)}
\end{Highlighting}
\end{Shaded}

\begin{verbatim}
## [1] 66
\end{verbatim}

\begin{Shaded}
\begin{Highlighting}[]
\FunctionTok{print}\NormalTok{(directed)}
\end{Highlighting}
\end{Shaded}

\begin{verbatim}
## [1] 132
\end{verbatim}

\hypertarget{uxbb38uxc81c-4}{%
\subsection{문제 4}\label{uxbb38uxc81c-4}}

\begin{Shaded}
\begin{Highlighting}[]
\CommentTok{\# q4}

\NormalTok{n }\OtherTok{=} \DecValTok{12}
\NormalTok{undirected }\OtherTok{=}\NormalTok{ n }\SpecialCharTok{*}\NormalTok{ (n }\SpecialCharTok{{-}} \DecValTok{1}\NormalTok{) }\SpecialCharTok{/} \DecValTok{2}
\NormalTok{link }\OtherTok{=} \DecValTok{20}
\NormalTok{density }\OtherTok{=}\NormalTok{ link }\SpecialCharTok{/}\NormalTok{ undirected}
\FunctionTok{print}\NormalTok{(density)}
\end{Highlighting}
\end{Shaded}

\begin{verbatim}
## [1] 0.3030303
\end{verbatim}

\hypertarget{uxb3c4uxc804uxbb38uxc81c}{%
\subsection{도전문제}\label{uxb3c4uxc804uxbb38uxc81c}}

\begin{Shaded}
\begin{Highlighting}[]
\CommentTok{\# dojeon}

\NormalTok{n }\OtherTok{=} \DecValTok{12}
\NormalTok{undirected }\OtherTok{=}\NormalTok{ n }\SpecialCharTok{*}\NormalTok{ (n }\SpecialCharTok{{-}} \DecValTok{1}\NormalTok{) }\SpecialCharTok{/} \DecValTok{2}
\NormalTok{all\_networks }\OtherTok{=} \DecValTok{2} \SpecialCharTok{\^{}}\NormalTok{ undirected }\CommentTok{\# 링크 각자가 끊어지기도 연결되기도 하니까 링크 당 경우의 수 "2"로 계산해서 2\^{}(링크 수)}
\NormalTok{all\_networks }\CommentTok{\# 출력}
\end{Highlighting}
\end{Shaded}

\begin{verbatim}
## [1] 7.378698e+19
\end{verbatim}

\end{document}
